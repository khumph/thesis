\RequirePackage[l2tabu, orthodox]{nag}
\documentclass[handout]{beamer}
% \documentclass{beamer}
\usetheme{metropolis}
\usepackage{appendixnumberbeamer}

\usepackage{booktabs}
\usepackage[scale=2]{ccicons}

% makes table of contents and such hyperlinks
\usepackage{hyperref}

\usepackage[style=authoryear, hyperref = true]{biblatex}
\addbibresource{library.bib}

\setbeamercolor{background canvas}{bg=white}
\newcommand\blfootnote[1]{%
  \begingroup
  \renewcommand\thefootnote{}\footnote{#1}%
  \addtocounter{footnote}{-1}%
  \endgroup
}

\setlength{\leftmargini}{12pt}

\usepackage{xspace}
\usepackage{commath}
\usepackage{graphics}
\usepackage{algorithm2e}
\DeclareMathOperator*{\argmax}{argmax}

\title{Using reinforcement learning to personalize dosing strategies in a simulated cancer trial with high dimensional data}
\subtitle{\sl or: how I learned to stop worrying and love creating billion-observation data frames}
\date{\today}
\author{Kyle Humphrey}
\institute{University of Arizona}
% \titlegraphic{\hfill\includegraphics[height=1.5cm]{logo.pdf}}

\begin{document}

\maketitle

\begin{frame}[c]{Table of contents}
  \setbeamertemplate{section in toc}[sections numbered]
  \tableofcontents
\end{frame}

\section{Introduction}

\begin{frame}[c]{Personalized medicine}

Patients often show significant differences in their responses to treatment
\begin{itemize}[<+(1)->]
  \item Adverse drug reactions alone are a significant burden 
  \begin{itemize}
    \item Accounted for 6.5\% of hospital admissions at two hospitals in England  (\cite{Pirmohamed2004})
    \item Projected \$850 million cost to NHS annually 
  \end{itemize}
  \item Other factors increase patient response to treatment
  \begin{itemize}
    \item Whether metastatic breast cancer cells over-express human epidermal growth factor receptor 2 (are HER2-positive)\footnotemark 
  \end{itemize}
\end{itemize}

\only<5->{\footnotetext{Usually, ``personalized medicine'' refers to this type of biomarker-based personalization}}

\end{frame}

\begin{frame}[c]{How to personalized medicine?}

Typically requires identifying patient subgroups with differential treatment responses
\begin{itemize}[<+(1)->]
  \item Statistically, has traditionally required treatment by covariate interactions
  \begin{itemize}
    \item Often there are lots of potential subgroups (lots of interactions) - need lots of data and method to avoid inflating type I error rate
  \end{itemize}
  \item Lots of novel methods have been proposed
  \begin{itemize}
    \item Problem (for our purposes) treatments always categorical (usually binary), only a single stage considered
  \end{itemize}
\end{itemize}

\end{frame}

\begin{frame}[c]{Broad overview}

Goal: find sequence of doses of a single treatment that maximize survival time in a simulated cancer clinical trial\footnote{setup inspired by \textcite{crt}} with and without high dimensional covariates and patient subgroups
\begin{itemize}[<+(1)->]
    \item Use Q-learning, a technique from reinforcement learning
    \item Use flexible methods from statistical learning to fit Q-functions: 
    \begin{itemize}[<+->]
      \item Regression trees (CART--\cite{CART})
      \item Multivariate adaptive regression splines (MARS--\cite{mars})
      \item Random forest (RF--\cite{rf})
    \end{itemize} 
\end{itemize}

\end{frame}
%--- Next Frame ---%

\section{Reinforcement learning} % (fold)
\label{sec:reinforcement_learning}

\begin{frame}[c]{Reinforcement learning: sequential decision making}
  \begin{figure}
    \centering
    \includegraphics[width = 0.6\textwidth]{figure/rl-venn-silver}\blfootnote{\scriptsize Image by \href{http://www0.cs.ucl.ac.uk/staff/D.Silver/web/Teaching.html}{David Silver}}
  \end{figure}
\end{frame}
%--- Next Frame ---%

\begin{frame}[c]{Reinforcement learning: basic elements}

\begin{itemize}[<+->]
  \item learning \emph{agent} who has goals related to its
  \item \emph{environment} (everything outside the agent's direct control)
  \begin{itemize}
    \item Features of the environment are represented to the agent through \emph{states}
  \end{itemize}
  \item The agent follows a \emph{policy} which dictates what 
  \item \emph{action} to take given a state signal to affect the environment in order to achieve some goal
  \begin{itemize}
    \item goal defined to agent so that maximizing a scalar quantity called the \emph{reward} would achieve the goal
  \end{itemize}
  \item Goal of reinforcement learning to find an \emph{optimal policy}: a policy that achieves the largest reward over the long run 
\end{itemize}

\end{frame}
%--- Next Frame ---%

\begin{frame}[c]{Reinforcement learning: basic process}

The agent:
\begin{enumerate}[<+->]
  \item is presented with states
  \item tries a sequence of actions
  \item records the consequences (rewards)
  \item statistically estimates the relationship between actions, states, and rewards
  \item chooses next action that leads to largest rewards
  \begin{itemize}
    \item Actions with large rewards tend to be repeated, hence ``reinforcement''
  \end{itemize}
\end{enumerate}
\end{frame}
%--- Next Frame ---%

% \begin{frame}[c]{Reinforcement learning: example}
%
% Board games offer quintessential example, take chess:
% \begin{itemize}
%     \item States: the positions of the pieces on the board
%     \item Actions: any of the legal moves with any given piece
%     \item Rewards: e.g. \\
%     if win $reward \gets 1$ \\
%     if lose $reward \gets 0$
%     \item Policy: ``strategy''
% \end{itemize}
%
% AlphaGo uses reinforcement learning and neural networks
%
% \end{frame}
% %--- Next Frame ---%

\begin{frame}[c]{Reinforcement learning: relationship to other areas of machine learning}
\begin{itemize}[<+->]
  \item Supervised learning
  \begin{itemize}
    \item learning agent given a list of correct example actions to taken in given situations
    \item goal: agent extrapolates correct example behavior to new situations
  \end{itemize}
  \item Unsupervised learning
  \begin{itemize}
    \item learning agent given a list of example actions and situations where the correct action is unknown
    \item goal: agent finds hidden structure
  \end{itemize} 
  \item Reinforcement learning
\begin{itemize}
  \item No correct example actions required
  \item goal: maximize rewards, not find hidden structure
  \item Feedback is delayed 
  \item Problems are closed loop: actions taken now affect available actions and rewards later
\end{itemize} 
\end{itemize}
\end{frame}
%--- Next Frame ---%


\begin{frame}[c]{Reinforcement learning: connection to personalized medicine}
  
  \begin{itemize}[<+->]
      \item States: patient histories and characteristics
      \item Actions: possible treatment options
      \item Rewards: goal dependent (e.g. kg weight lost in weight loss study)
      \item Policy: \emph{dynamic treatment regime} 
  \end{itemize}

\end{frame}
%--- Next Frame ---%

\subsection{Q-learning} % (fold)
\label{sub:q_learning}



\begin{frame}{Q-learning: setup}
  Patients recruited into a trial,
  \begin{enumerate}[<+(1)->]
    \item covariates measured, stored in $O_{1}$; defined as patient history at stage 1: $H_{1} \equiv O_{1}$
    \item treatment, $A_{1}$ randomly assigned
    \item covariates measured again, stored in $O_{2}$, reward $R_{1}$ received, everything recorded into patient history at stage 2: $H_{2} \equiv (O_{1}, A_{1}, O_{2})$
    \item second treatment $A_{2}$ randomly assigned (could depend on $H_{2}$)
    \item covariates measured again ($O_{3}$) and another reward $R_{2}$ received, $H_{3} \equiv (H_{2}, A_{2}, O_{3})$, process repeated over $K$ stages
  \end{enumerate}
    \pause
Q-function at stage $k$ defined as:
  \begin{equation*}
     Q_{k}(A_{k}, H_{k})  = \operatorname{E}[r_{k} \mid A_{k}, H_{k}], \quad k = 1, \ldots, K
  \end{equation*}
\end{frame}
%--- Next Frame ---%

\begin{frame}[c]{Q-learning}
\begin{algorithm}[H]
  $\hat{Q}_{K + 1} \gets 0$

  \medskip

 \For{$k$ in $K, \ldots, 1$}{
  Estimate rewards that would occur if optimal policy followed after $k$:

$ \hat{r}_{k} \gets r_{k} + \max_{A_{k+1}} \hat{Q}_{k+1}(A_{k+1}, H_{k+1})$

\medskip

Estimate Q-function, $Q_{k}(A_{k}, H_{k})$, for preceding stage using this reward as outcome (e.g. with ordinary least squares).

\medskip

Estimate the optimal treatment for each patient at stage $k$:

$\hat{A}^{opt}_{k} \gets \argmax_{A_{k}} \hat{Q}_{k}(A_{k}, H_{k})$
}
\end{algorithm}
\end{frame}
%--- Next Frame ---%

% subsection q_learning (end)

% section reinforcement_learning (end)

\section{Methods to fit Q-functions} % (fold)
\label{sec:methods_to_fit_q_functions}

\begin{frame}[c]{Why trees (and MARS)?}
  \begin{itemize}[<+->]
    \item ``Automatically'' conduct variable selection and interaction modelling
    \item Not restricted to linear relationships between $X$ and $Y$
    \item Produces interpretable models 
  \end{itemize}
\end{frame}
%--- Next Frame ---%

\subsection{Regression trees (CART)} % (fold)
\label{sub:cart}

\begin{frame}{Regression trees (CART): example}
  For 50 patients with initial tumor mass $M_{0} \overset{iid}{\sim} \text{Unif}(1, 2)$,
  \begin{itemize}[<+->]
    \item try to predict change in tumor mass: $Y_{i} \equiv M_{i1} - M_{i0}$
    \item given what dose of treatment $D_{i}$ is assigned
    \item true relationship:
    \begin{align*}
      Y_{i} &= 0.75 - 1.2 D_{i} + \epsilon_{i}, \quad i = 1, \ldots, 50 \\
      & \epsilon_{i} \sim N(0, 0.05)
    \end{align*}
     
  \end{itemize}
  
\end{frame}
%--- Next Frame ---%

\begin{frame}[c]{Regression trees (CART): example tree}
  \begin{figure}[!htbp]
  \begin{center}
    \includegraphics[width=\textwidth]{figure/ex-cart-tree-1}
  \end{center}
  \end{figure}
\end{frame}
%--- Next Frame ---%

\begin{frame}{Regression trees (CART): tree growing}
  Goal: make nodes as homogeneous in outcome as possible
  
  \pause
  Need to decide:
  \begin{enumerate}[<+->]
    \item what covariate to use to define a split and what value of the covariate to split on
    \item when to stop splitting
    \item how to assign an outcome to each terminal node
  \end{enumerate}
\end{frame}
%--- Next Frame ---%

\begin{frame}{Regression trees (CART): tree growing}
  Start with root node
  
  \begin{enumerate}[<+(1)->]
  \item Try splitting each unique value $x_{ij}$ of each covariate $X_{j}$
  \item calculate the resulting $SSE$
  \begin{equation*}
    SSE = \sum_{i \in \text{node}_{L}} (y_{i} - \bar{y}_{L})^2 + \sum_{i \in \text{node}_{R}} (y_{i} - \bar{y}_{R})^2
  \end{equation*}
  $\bar{y}_{L}$ is the average outcome in the left node, $\text{node}_{L}$, and $\bar{y}_{R}$ is the average outcome in the right node, $\text{node}_{R}$.
  \item Pick split that minimizes $SSE$
    \end{enumerate}
  \pause
  Repeat for resulting nodes 
    
  \begin{itemize}[<+(1)->]
  \item Stop splitting once e.g. resulting nodes would be too small
  \item Assign average outcome in each terminal node as predicted outcome 
  \end{itemize}
\end{frame}
%--- Next Frame ---%

\begin{frame}[c]{Regression trees (CART): tree pruning}
  \begin{itemize}[<+->]
    \item Tree growing algorithm usually produces a bushy/deep tree $T_{0}$ that overfits
    \item So use cost-complexity pruning: goal to find subtrees $T \subset T_{0}$ which minimize
    \begin{equation*}
        C_{\alpha}(T) = \sum_{m = 1}^{\abs{T}} \sum_{x_{i} \in \text{node}_{m}} (y_{i} - \bar{y}_{m})^2 + \alpha \abs{T}
    \end{equation*}
    \item Unique subtree $T_{\alpha}$ that minimizes $C_{\alpha}(T)$ for each $\alpha \geq 0$.
  \end{itemize}
\end{frame}
%--- Next Frame ---%

\begin{frame}[c]{Regression trees (CART): tree pruning}
  To find $T_{\alpha}$, use weakest link pruning
      \begin{enumerate}[<+->]
        \item Try collapsing each internal node in turn
        \item actually collapse weakest link (node that if collapsed produces smallest increase in $SSE$), save tree
        \item repeat until you're at root node 
          \pause
          
          in results: sequence of $T_{\alpha}$s  
        \item choose best $\alpha$/$T_{\alpha}$ from sequence using resampling technique (e.g. 10-fold cross validation)
      \end{enumerate}
\end{frame}
%--- Next Frame ---%

\begin{frame}[c]{Regression trees (CART): example tree}
  \begin{figure}[!htbp]
  \begin{center}
    \includegraphics[width=\textwidth]{figure/ex-cart-tree-1}
  \end{center}
  \end{figure}
\end{frame}
%--- Next Frame ---%

\begin{frame}{Regression trees (CART) example function}
  \begin{figure}[!htbp]
  \begin{center}
    \includegraphics[width=0.9\textwidth]{figure/ex-plot-cart-1}
  \end{center}
  \end{figure}
\end{frame}
%--- Next Frame ---%

% \begin{frame}{Regression trees (CART)}
%   Why we like trees:
%   \begin{enumerate}
%     \item Easy to interpret
%     \item Feature selection and modelling of interactions part of process
%     \item Easy to handle many types of covariates without pre-processing
%   \end{enumerate}
%
%   Why we don't like trees:
%   \begin{enumerate}
%     \item Generally have lower performance than other methods, particularly in regression setting
%   \end{enumerate}
% \end{frame}
% %--- Next Frame ---%

% subsection cart (end)

\subsection{MARS} % (fold)
\label{ssub:mars}

\begin{frame}{Multivariate Adaptive Regression Splintes}
  \begin{itemize}[<+->]
    \item Generalization of stepwise linear regression methods that incorporates modelling interactions
    \item achieved by modelling covariates as piecewise linear basis functions and their products:
  %
  \begin{equation*} \label{eq:bases}
    (x - t)_{+} \text{ and } (t - x)_{+}
  \end{equation*} where
  %
  \begin{equation*}
    (Z)_{+} = \max(Z, 0) = \begin{cases}
    Z, & \text{if } Z > 0 \\
    0, & \text{otherwise}
    \end{cases}
  \end{equation*}
  
  \item Each half of the pair is called a \emph{hinge function}, or linear spline
  \item $t$ is a \emph{knot}
  \end{itemize}
\end{frame}
%--- Next Frame ---%

\begin{frame}{MARS: Model building (forward pass)}
  
  \begin{itemize}[<+->]
    \item Like in forward selection, start with a constant:
    \begin{equation*}
      f(X) = \beta_{0}
    \end{equation*}
    \item Consider adding pairs of piecewise linear functions:
  \begin{equation*}
  \beta_{1} (X_{j} - x_{ij})_{+} + \beta_{2}(x_{ij} - X_{j})_{+}, \ i = 1, 2, \ldots, n; \ j = 1, 2, \ldots, p.
  \end{equation*}
   made by splitting some covariate $X_{j}$ at one of the observed values of $X_{j}$, $x_{ij}$
  \begin{enumerate}
    \item Try adding each possible pair to the model
    \item calculate the $SSE$
    \item add the pair that minimizes $SSE$
  \end{enumerate}
  \end{itemize}

\end{frame}
%--- Next Frame ---%

\begin{frame}{MARS: Model building (forward pass)}
  \begin{itemize}[<+->]
    \item Then consider adding pairs with general form:
  \begin{equation*} \label{eq:gen-pairs-add}
    \beta_{M + 1} h_{\ell}(X) \cdot (X_{j} - x_{ij})_{+} + \beta_{M + 2} h_{\ell}(X) \cdot (x_{ij} - X_{j})_{+}
  \end{equation*}
  $h_{\ell}(X)$ is a hinge function (or product of hinge functions) already in the model\footnote{Also holds for the first step where $h_{\ell}(X) = h_{0}(X) = 1$}
  \item As before, add pair that minimizes $SSE$
  \item Repeat until 
  \begin{itemize}
    \item change in $SSE$ too small and/or
    \item upper limit on number of terms in model reached
  \end{itemize}
  \end{itemize}
\end{frame}
%--- Next Frame ---%

\begin{frame}{MARS: Backward pass}
  \begin{itemize}[<+->]
    \item Resulting model usually overfits
    \item So like in backward selection, remove terms from model one by one:
    \begin{enumerate}
      \item Estimate how much $SSE$ would increase by removing each term in turn
      \item Remove term that increases $SSE$ the least
      \item Repeat until no terms left
    \end{enumerate}
    \item At each stage, we obtain $\hat{f}_{\lambda}(X)$, best model with $\lambda$ terms
  \end{itemize}

\end{frame}
%--- Next Frame ---%

\begin{frame}{MARS: Backward pass}
Pick $\lambda$ (and hence final model) using generalized cross validation (GCV) 
    \begin{equation*}
    GCV(\lambda) =
      \frac{
        \sum_{i = 1}^{N}(y_{i} - \hat{f}_{\lambda}(x_{i}))^2
      }{
        (1 - M(\lambda)/N)^2
      }
  \end{equation*}
  $M(\lambda) = \lambda + 3K$ (number of terms plus three times the number of knots)
  \begin{itemize}
    \item<+-> can be set to $\lambda + 2K$ when degree = 1 (restricted to be additive)
  \end{itemize}

\end{frame}
%--- Next Frame ---%

\begin{frame}{MARS: Relationship to CART}
  
  MARS can also be viewed as a modification of CART to improve performance in the regression setting.
  
  MARS forward pass is equivalent to CART tree growing algorithm if we:
  \begin{enumerate}
    \item Change hinge functions to step functions:
    \begin{equation*}
      I(x - t > 0) \text{ and } I(t - x > 0)
    \end{equation*}
    \item Replace terms already in the model involved in a new interaction term by the new interaction term (making original term unavailable for future interactions)
  \end{enumerate}
\end{frame}
%--- Next Frame ---%


\begin{frame}[t]{MARS: example}

\begin{figure}[!htbp]
\begin{center}
  \includegraphics[width=0.9\textwidth]{figure/ex-plot-mars-1}
\end{center}
\end{figure}
%
\begin{equation*}
  \label{eq:mars-eqn}
  \hat{f}(D) \approx -0.24 + 1.24 (0.81 - D)_{+} -1.64 (D - 0.87)_{+}
\end{equation*}

\end{frame}
%--- Next Frame ---%

% subsubsection mars (end)

\subsection{Bagging and Random Forest} % (fold)
\label{sub:bagging_and_random}

\begin{frame}{Bagging}
  
  \begin{itemize}
    \item Single trees tend to have quite high variance (small changes in the data can produce very different trees)
    \item When grown sufficiently deep or bushy, they can have relatively low bias 
  \end{itemize}
  
  Bootstrap aggregating (\emph{bagging}) is a general technique that is useful whenever a modelling technique has high variance and low bias.
  
\end{frame}
%--- Next Frame ---%

\begin{frame}{Bagging}
  
  \begin{itemize}[<+->]
    \item Consider iid random variables, $X_{1}, X_{2}, \ldots, X_{n}$, each with variance $\sigma^2$ and sample mean $\bar{X}$.
  \begin{equation*}
    \operatorname{Var}[\bar{X}] = \frac{\sigma^2}{n}
  \end{equation*}
  \item Thus, can improve tree performance by building trees on many different samples then averaging results
  \item But we don't have many different samples (only have one), so
  \begin{enumerate}
    \item take $B$ bootstrap samples of sample
    \item fit a model to each sample (grow a tree, don't prune)
    \item average the resulted predictions 
  \end{enumerate}
  \end{itemize}
  
\end{frame}
%--- Next Frame ---%

\begin{frame}{Random forest}
  
  \begin{itemize}[<+->]
    \item Can't in practice achieve a $B$ fold decrease in the variance from a single tree because trees are correlated
    \item \textcite{rf}: decrease correlation by considering a random subset of $m_{try} \subset p$ predictors for splitting at each split
    \item Choose $m_{try}$ using resampling techniques (e.g. cross validation)
    \item Requires more resamples than bagging (due to introduced randomness)
    \item Performance much better than single tree, but also much less interpretable 
  \end{itemize}
\end{frame}
%--- Next Frame ---%

\begin{frame}{Random forest (bagging) example}
  
  \begin{figure}[!htb]
  \begin{center}
    \includegraphics[width=0.9\textwidth]{figure/ex-plot-rf-1}
  \end{center}
  \end{figure}
  
\end{frame}
%--- Next Frame ---%

% subsection bagging_and_random (end)

% section methods_to_fit_q_functions (end)

\section{Simulation} % (fold)
\label{sec:simulation}

\begin{frame}[c]{Setup}
  Generated three random doses for 1000 patients each dose between none and maximum tolerable:
  \begin{align*}
    D_{t} &\overset{iid}{\sim} \text{Unif}(0, 1), \quad t = 0, 1, 2
  \end{align*}
  Each patient starts with random initial values of toxicity ($W$)\footnote{For simplicity, I take wellness and quality of life to be equivalent and exactly the opposite of toxicity} and tumor mass ($M$) over $(0,2)$
  \begin{align*}
    W_{0} &\overset{iid}{\sim} \text{Unif}(0, 2) \\
    M_{0} &\overset{iid}{\sim} \text{Unif}(0, 2)
  \end{align*}
\end{frame}
%--- Next Frame ---%

\begin{frame}[c]{Patient model: transition functions}
  
  Toxicity:
  \begin{equation*}
  W^{*}_{t} = 0.1 M_{t-1} + 1.2 (D_{t-1} - 0.5) + W_{t - 1}
  \end{equation*}
  \begin{equation*}
  W_{t} = \begin{cases}
    W^{*}_{t} &\text{if } W^{*}_{t} > 0 \\
    0 &\text{if } W^{*}_{t} \leq 0
  \end{cases}
  \end{equation*}
  
  Tumor mass:
  \begin{equation*}
  M^{*}_{t} = [0.15 W_{t-1} - 1.2 (D_{t-1} - 0.5) + M_{t - 1}] I(M_{t-1} > 0)
  \end{equation*}
  \begin{equation*}
  M_{t} = \begin{cases}
    M^{*}_{t} &\text{if } M^{*}_{t} > 0 \\
    0 &\text{if } M^{*}_{t} \leq 0
  \end{cases}
  \end{equation*}
  
\end{frame}
%--- Next Frame ---%

\begin{frame}[c]{Patient model: survival}
  \begin{align*}
    S^{*}_{it} &\sim \operatorname{exponential}(\beta_{it}(M_{it + 1}, W_{it + 1})) \\[1em]
    \beta_{it}(M_{it+1}, W_{it+1}) &= \exp(5.5 - W_{it+1} - 1.2 M_{it+1} - 0.75 W_{it+1} M_{it+1}) \\[1em]
    S_{it} &= \begin{cases}
      S^{*}_{it} + \sum_{j = 0}^{t} j & \text{if } S_{it} \leq 1 \text{ or } t = 2 \\
      \text{undefined} & \text{otherwise}
    \end{cases} \\[1em]
    R_{it + 1} &=\begin{cases}
          \log(S_{it}) & \text{if } S_{it} \leq 1 \text{ or } t = 2 \\
          0 & \text{otherwise\footnotemark}
        \end{cases}
  \end{align*}  
  \footnotetext{log of the survival time taken as continuous reward (and hence outcome), not as a censored time-to-event outcome}
\end{frame}
%--- Next Frame ---%

\begin{frame}[c]{Example}
  \includegraphics[width=\textwidth]{figure/ind-plot-1} 
\end{frame}
%--- Next Frame ---%

\begin{frame}[c]{Interaction (I) scenario}
  Two additional baseline covariates for each patient:
  \begin{equation*}
    X_{1}, X_{2} \overset{iid}{\sim} \text{Unif}(0, 1)
  \end{equation*}
  
  $X_{1} < 0.5 \ \& \ X_{2} < 0.5$: same as before
  
  
  $X_{1} > 0.5 \ \& \ X_{2} < 0.5$: 50\% more sensitive to drug effects:
  \begin{equation*}
  W^{*}_{t} = W'_{t} = 0.1 M_{t-1} + 1.2 (1.5 D_{t-1} - 0.5) + W_{t - 1}
  \end{equation*}
  %
  $X_{1} < 0.5 \ \& \ X_{2} > 0.5$: drug 50\% more effective:
  \begin{equation*}
  M^{*}_{t} =  M'_{t} = [0.15 W_{t-1} - 1.2 (1.5 D_{t-1} - 0.5) + M_{t - 1}] I(M_{t-1} > 0)
  \end{equation*}
  % 
  $X_{1} > 0.5 \ \& \ X_{2} > 0.5$: 50\% more effective and 50\% more sensitive:
  \begin{align*}
  W^{*}_{t} = W'_{t} &= 0.1 M_{t-1} + 1.2 (1.5 D_{t-1} - 0.5) + W_{t - 1} \\
  M^{*}_{t} = M'_{t} &= [0.15 W_{t-1} - 1.2 (1.5 D_{t-1} - 0.5) + M_{t - 1}] I(M_{t-1} > 0)
  \end{align*}
\end{frame}
%--- Next Frame ---%

\begin{frame}[c]{Noise scenarios}
  
  Noise (N) scenario: 100 additional baseline covariates with no effect:
  \begin{align*}
    Z_{1}, \ldots, Z_{5} &\overset{iid}{\sim} N(1, 1) \\
    Z_{6}, \ldots, Z_{10} &\overset{iid}{\sim} N(-1, 1) \\
    V_{1}, \ldots, V_{90} &\overset{iid}{\sim} N(0, 1)
  \end{align*}
  
  Predictive noise (PN) scenario: same 100 baseline covariates, but
  %
  \begin{align*}
    \beta'_{it}&(M_{it+1}, W_{it+1}) = \\
    &\exp\del{5.5 + W_{it+1} + 1.2 M_{it+1} + 0.75 W_{it+1} M_{it+1} + 0.05 \sum_{j = 1}^{10} Z_{ij}}
  \end{align*}
\end{frame}
%--- Next Frame ---%

\begin{frame}[c]{Interaction + noise scenarios}
  Interaction + noise scenario (I+N):
  
  four subgroups as in interaction scenario plus the 100 noise variables from noise scenario
  
  \bigskip
  
  Interaction + predictive noise scenario (I+PN):
  
  four subgroups as in interaction scenario plus the 100 noise variables from noise scenario, but $Z$ variables shift expected survival up or down
\end{frame}
%--- Next Frame ---%

% section simulation (end)

\begin{frame}[c]{Training procedure}
  
  Q-learning applied using:
  \begin{itemize}
    \item CART: trees that minimized 10 fold cross validation error\footnote{\label{tree} Splits creating terminal nodes with $\leq$ 5 observations not considered}
    \item MARS: restricted to second degree interactions with $D$ only
    \item Random forest: bagging turned out to be best under each scenario\textsuperscript{\ref{tree}}
  \end{itemize}
  
  Fitted models for each stage under each scenario saved for estimating optimal treatments for patients in the validation set
  
  Repeated 100 times with different simulated patients
\end{frame}
%--- Next Frame ---%

\begin{frame}[t]{Validation procedure}
  
  2000 new patients simulated identically to those in each scenario, copies of the 2000 underwent:
  \begin{itemize}[<+(1)->]
    \item 10 constant dose regimes: doses of $0.1, 0.2, \ldots,$ or 1 given at each stage
    \item Best: dose sequence out of all possible sequences with $D_{t} \in \{0, 0.01, 0.02, \ldots, 1\}$ that maximizes expected survival given complete knowledge of the transition functions and their influence on survival, as long as $S_{it} \leq 1$, $t = 0, 1 \enspace \forall i$.
    \item CART, MARS, RF: Doses from $\{0, 0.01, 0.02, \ldots, 1\}$, corresponding to the maximum predicted rewards for each of the 100 models for each stage
  \end{itemize}
\end{frame}
%--- Next Frame ---%

\section{Results} % (fold)
\label{sec:results}


\begin{frame}[c]{Survival times: scenarios without interaction}
\centering
  \includegraphics[width=0.98\textwidth]{figure/pres-results-no-int-1} 
\end{frame}
%--- Next Frame ---%

\begin{frame}[c]{Survival times: scenarios without interaction}
  \centering
  \includegraphics[width=0.98\textwidth]{figure/pres-results-int-1} 
\end{frame}
%--- Next Frame ---%

\begin{frame}[c]{Variable importance: scenarios without interaction}
\centering
  \includegraphics[width=0.95\textwidth]{figure/pres-var-imps-no-int-1} 
\end{frame}
%--- Next Frame ---%

\begin{frame}[c]{Variable importance: scenarios with interaction}
  \centering
  \includegraphics[width=0.95\textwidth]{figure/pres-var-imps-int-1} 
\end{frame}
%--- Next Frame ---%


\begin{frame}[t]{Discussion}
  \begin{itemize}[<+->]
    \item Q-learning using random forest (bagging) and MARS (only treatment interactions allowed) increased average survival substantially over constant doses
    \item Single regression tree (CART) did not
    \begin{itemize}
      \item Predictive accuracy was splitting and pruning criterion: dose and interacting variables (less important in overall survival) often pruned off 
    \end{itemize}
    \item MARS gives comparable performance to random forest, but has higher variance in both performance on test set and variable importances 
    \item As implemented subgroup identification is difficult, likely need to modify techniques to split upon magnitude of (qualitative) interaction with treatment 
  \end{itemize}
\end{frame}
%--- Next Frame ---%


% section results (end)

\begin{frame}[standout]
  Thanks
\end{frame}

\appendix

\begin{frame}[c]{Mean survival times}
  \begin{table}
  \small
  \centering
  \begin{tabular}{lrrrrrr}
  \toprule \multicolumn{1}{c}{Regime}&\multicolumn{1}{c}{B}&\multicolumn{1}{c}{N}&\multicolumn{1}{c}{PN}&\multicolumn{1}{c}{I}&\multicolumn{1}{c}{I+N}&\multicolumn{1}{c}{I+PN}\tabularnewline
  \midrule
  Best&$60.0$&$60.0$&$61.1$&$79.5$&$79.5$&$81.1$\tabularnewline
  MARS&$39.8$&$29.1$&$27.7$&$43.9$&$35.8$&$37.0$\tabularnewline
  RF&$36.3$&$30.0$&$30.7$&$45.8$&$41.2$&$42.1$\tabularnewline
  CART&$12.5$&$ 9.9$&$10.0$&$12.8$&$10.0$&$ 9.9$\tabularnewline
  1&$12.5$&$12.5$&$12.7$&$ 9.2$&$ 9.2$&$ 9.4$\tabularnewline
  0.9&$14.5$&$14.5$&$14.8$&$11.8$&$11.8$&$12.0$\tabularnewline
  0.8&$16.2$&$16.2$&$16.5$&$15.0$&$15.0$&$15.3$\tabularnewline
  0.7&$16.6$&$16.6$&$16.9$&$18.0$&$18.0$&$18.4$\tabularnewline
  0.6&$16.2$&$16.2$&$16.5$&$20.1$&$20.1$&$20.5$\tabularnewline
  0.5&$15.5$&$15.5$&$15.8$&$21.1$&$21.1$&$21.5$\tabularnewline
  0.4&$15.6$&$15.6$&$16.0$&$19.8$&$19.8$&$20.2$\tabularnewline
  0.3&$14.7$&$14.7$&$15.0$&$16.8$&$16.8$&$17.2$\tabularnewline
  0.2&$12.7$&$12.7$&$12.9$&$14.1$&$14.1$&$14.4$\tabularnewline
  0.1&$10.5$&$10.5$&$10.6$&$11.0$&$11.0$&$11.2$\tabularnewline
  \bottomrule
  \end{tabular}
  \end{table}
\end{frame}
%--- Next Frame ---%

\begin{frame}{Standard deviations of survival times across training sets}
  \begin{table}[!htbp]
  \centering
  \begin{tabular}{lrrrrrr}
  \toprule \multicolumn{1}{c}{Regime}&\multicolumn{1}{c}{B}&\multicolumn{1}{c}{N}&\multicolumn{1}{c}{PN}&\multicolumn{1}{c}{I}&\multicolumn{1}{c}{I+N}&\multicolumn{1}{c}{I+PN}\tabularnewline
  \midrule
  MARS&$9.8$&$8.8$&$8.3$&$9.6$&$7.9$&$9.0$\tabularnewline
  RF&$5.2$&$5.2$&$5.8$&$6.0$&$6.4$&$6.5$\tabularnewline
  CART&$3.0$&$2.5$&$2.2$&$4.0$&$3.2$&$2.7$\tabularnewline
  \bottomrule
  \end{tabular}
  \end{table}
\end{frame}
%--- Next Frame ---%

\begin{frame}[c]{}
  \begin{columns}
    \column{\dimexpr\paperwidth}
\begin{table}
\scriptsize
\centering
\begin{tabular}{lllllll}
\toprule
\multicolumn{1}{l}{Variable}&\multicolumn{1}{c}{B}&\multicolumn{1}{c}{N}&\multicolumn{1}{c}{PN}&\multicolumn{1}{c}{I}&\multicolumn{1}{c}{I+N}&\multicolumn{1}{c}{I+PN}\tabularnewline
\midrule
{\bfseries MARS}&&&&&&\tabularnewline
~~$M$&98.0 (6.5)&97.0 (7.7)&95.0 (9.5)&99.0 (4.0)&98.0 (6.6)&98.0 (6.8)\tabularnewline
~~$W$&78.0 (9.2)&83.0 (7.9)&84.0 (9.9)&82.0 (6.4)&86.0 (7.3)&85.0 (6.9)\tabularnewline
~~$D$&25.0 (12.4)&37.0 (8.4)&38.0 (8.4)&36.0 (13.1)&47.0 (11.3)&45.0 (9.6)\tabularnewline
~~$X$&&&&32.0 (12.4)&46.0 (12.3)&41.0 (14.3)\tabularnewline
~~$Z$&&6.0 (11.9)&8.0 (13.1)&&6.0 (12.4)&7.0 (13.2)\tabularnewline
~~$V$&&5.0 (11.0)&6.0 (11.2)&&6.0 (12.4)&6.0 (11.4)\tabularnewline
\midrule
{\bfseries RF}&&&&&&\tabularnewline
~~$M$&100.0 (0.6)&100.0 (1.2)&99.0 (3.3)&99.0 (2.1)&99.0 (3.1)&100.0 (1.8)\tabularnewline
~~$W$&78.0 (11.2)&80.0 (12.5)&81.0 (13.4)&86.0 (12.6)&87.0 (10.6)&85.0 (10.9)\tabularnewline
~~$D$&18.0 (7.4)&2.0 (0.7)&2.0 (0.7)&19.0 (6.7)&4.0 (1.9)&4.0 (2.2)\tabularnewline
~~$X$&&&&25.0 (6.6)&8.0 (3.4)&7.0 (2.7)\tabularnewline
~~$Z$&&1.0 (0.5)&1.0 (0.6)&&1.0 (0.8)&2.0 (0.8)\tabularnewline
~~$V$&&1.0 (0.4)&1.0 (0.4)&&1.0 (0.5)&1.0 (0.5)\tabularnewline
\midrule
{\bfseries CART}&&&&&&\tabularnewline
~~$M$&99.0 (4.0)&98.0 (4.4)&98.0 (5.3)&96.0 (6.4)&97.0 (8.5)&98.0 (4.8)\tabularnewline
~~$W$&76.0 (16.5)&77.0 (17.5)&78.0 (15.6)&87.0 (13.9)&82.0 (16.2)&84.0 (14.9)\tabularnewline
~~$D$&7.0 (3.9)&6.0 (3.2)&6.0 (3.3)&8.0 (5.2)&7.0 (3.2)&7.0 (3.3)\tabularnewline
~~$X$&&&&9.0 (5.0)&6.0 (2.3)&6.0 (2.8)\tabularnewline
~~$Z$&&4.0 (2.0)&2.0 (1.3)&&3.0 (1.7)&4.0 (1.2)\tabularnewline
~~$V$&&3.0 (1.7)&3.0 (1.6)&&4.0 (1.8)&4.0 (2.1)\tabularnewline
\bottomrule
\end{tabular}
\end{table}
  \end{columns}
\end{frame}
%--- Next Frame ---%



\begin{frame}[allowframebreaks]{References}

\printbibliography[heading=none]

\end{frame}

\end{document}
