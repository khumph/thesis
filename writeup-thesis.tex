\documentclass[12pt]{article}
\usepackage[margin = 1in]{geometry}

% makes table of contents and such hyperlinks
\usepackage{hyperref}

% nicely formatted tables
\usepackage{booktabs}

% H to force figures and tables to where they appear in text
\usepackage{float}

% English language/hyphenation
\usepackage[english]{babel}

% > < print corectly, accented words hyphenate
\usepackage[utf8]{inputenc}
\usepackage[T1]{fontenc}

% allows usage of \enquote to quote things
\usepackage{csquotes}

% \usepackage[backend = bibtex, sorting=none, hyperref = true]{biblatex}
% \addbibresource{spray.bib}
\usepackage{cite}
\usepackage{ctable}

\usepackage{amsmath}

%--------------------------------------------------------------------------
%  TITLE SECTION
%--------------------------------------------------------------------------

\title{\normalfont \Large The Effects of Disinfection Products and Procedures on Spread of Virus in an Outpatient Clinic}

\author{\normalsize \sl Kyle Humphrey}

\date{\normalsize \sl \today}

\begin{document}

\maketitle

\section{Introduction}

\section{Reinforcement Clinical Trial}


I began by simulating 1000 patients over six months of treatment, each with a random starting tumor mass generated from a uniform distribution between 0 and 2 and a random starting toxicity between 0 and 2 also generated from a uniform distribution between 0 and 2. The initial does was picked from a uniform random distribution between 0.5 and 1, and subsequent month doses were randomly selected from a uniform distribution between 0 and 1.

Toxicity responded to dose as follows:

\[
\dot{Tox} = 0.1 M + 1.2 (D - 0.5)
\]

Where $\dot{Tox}$ is the change in toxicity from one month to another, 
$M$ is the tumor mass the previous month
$D$ is the does for the previous month

Tumor mass responded to dose as follows:

\[
\dot{M} = [0.15 Tox - 1.2 (D - 0.5)] I_{\{M > 0\}}(M)
\]

Where $\dot{M}$ is the change in tumor mass from one month to another,
$Tox$ is the toxicity from the previous month 
$M$ is the tumor mass the previous month
$D$ is the does for the previous month
$I_{\{M > 0\}}(M)$ indicates that if a person is cured (tumor mass = 0), then tumors do not recur


Rewards were defined as follows:

\[
R_{1}(M, Tox, D) = 
\begin{cases}
  -60 & \text{patient died} \\
  0 & \text{otherwise}
\end{cases}
\]

\[
R_{2}(M, Tox) = 
\begin{cases}
  -5 & \dot{W} > 0.5 \\
  0 & -0.5 \leq \dot{W} \leq 0.5 \\
  5 & \dot{W} < -0.5 \\
  15 & \dot{W} + W \leq 0
\end{cases}
\]

\[
R_{3}(M, Tox) = 
\begin{cases}
  -5 & \dot{M} > 0.5 \\
  0 & -0.5 \leq \dot{M} \leq 0.5 \\
  5 & \dot{M} < -0.5
\end{cases}
\]

\[
\lambda = -7 + W + M
\]

\[
\Delta F = e^{-\lambda}
\]

\[
p = 1 - \Delta F
\]

\[
\text{patient died} \sim B(p)
\]





\end{document}
